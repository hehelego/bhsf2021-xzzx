\documentclass[10pt]{beamer}
\usetheme{Madrid}
\usecolortheme{default}

\usepackage{hyperref}
\usepackage{xeCJK}
\usepackage{ulem}
\usepackage{geometry}
\usepackage{graphicx}
\setCJKmathfont{Noto Serif CJK SC}
\setCJKsansfont{Noto Sans CJK SC}


\title{杂谈:平常心}
\subsection{q}
\author{hehelego/spinach @ 张江理工}
\date{\today}



\begin{document}
\begin{frame}
	\titlepage{}
	\flushright{
		\fbox{
			\href{https://github.com/hehelego/bhsf2021-xzzx}{provided under the CC0 license, available on github.}
		}
	}
\end{frame}

\begin{frame}
	\frametitle{Table of Contents}
	\begin{block}{Skip it}
		\noindent ``想看书的人,不会去翻目录''\\
		\noindent keywords: 对现状/过程/理想的清醒认知, 说服自己, 内心的平静.
	\end{block}
	\pause{}
	\begin{itemize}
		\item 提示:\,体系的完整性
		\item ``平稳发挥''
			\begin{itemize}
					\item 为何追求平稳发挥
					\item 养成习惯\ 保持节奏
					\item 考试结果的正确打开方式
			\end{itemize}
		\item 一些小问题
		\item 那些不算问题的心态问题
			\begin{itemize}
					\item 尊重规律和极限
					\item 排除内在的\ 急和懒
					\item 你仍然是完整的人
			\end{itemize}
		\item 结语 \& 赠言
	\end{itemize}
\end{frame}




\begin{frame}
	\frametitle{一点复习备考心得}
	\begin{block}{抓住核心与关联:\,彻底解决记不起,想不到}
		\emph{\tiny{适用数学、物理、化学、生物}}\\
		主要模块 |  模块核心知识 | 模块的逻辑\\
		基础知识 + 主要技巧 + 印象\\
	\end{block}

	知识细节,特殊形式识别和技巧性解答当然重要,\\
	但不能失去全局观,忘记知识的连接.\\
	完整体系有助于形成可靠的记忆,并帮你对抗突破常规的题目.\\

	\pause{}

	\begin{example}{一些值得做的练习}
		\begin{itemize}
			\item 问问自己\ 物理学了哪几个模块?\\
				能否快速回忆出模块中最基础的几个公式是从哪个实验、公理导出的?\\
				题目中最长见的场景是什么样子的? 什么关系是有用的? 快速推导一下.\\
			\item 找找之前做过讲过的推导比较复杂的生物题,\\ 从课本上相关章节最基础的知识开始给自己讲清常楚它.\\
				(高考题是最合适的,出得比较严谨)
			\item 当然我相信这种练习各个学科教师在试卷讲评时都有做过.
		\end{itemize}
	\end{example}
\end{frame}


\begin{frame}
	\frametitle{平稳发挥}
	\begin{alertblock}{之后没有干货了}
		我不打算分享更多学习方法,或者学科技巧\quad
		``前人之述备矣''.
	\end{alertblock}
	\begin{block}{quotation}
		\flushleft{我从来不会说, 四中的同学们超常发挥了, 也不会希望你们超常发挥.
		只会说, 今年,他们考得很正常.}\\
		\flushright{from \sout{因为科老师退休而转正的}马校长}\\
		{\ }
		\flushleft{不仅考察知识是否掌握得牢固,解题技巧是否熟练,思维是否全面且深刻,
		更是考察心态是否平稳,是否能抵抗波动.}\\
		\flushright{from \sout{经常帮助翘课的孩子的}高杰老师}
	\end{block}
\end{frame}

\begin{frame}
	\frametitle{平稳发挥\ Cont'd\qquad 为什么我不能是黑马}
	你是否听过这些话语

	\begin{quote}
		完蛋了,虽然切了压轴但是挂了两个选择题, 垫底了.\\
		考场作文不必写出新意, 能用朴实的语言写出完整的叙事或者论证结构, 就可以得到不错的分数\\
		考前再次回归课本, 查缺补漏. 以求高考平稳发挥.\\
		同学们一定要保持状态,我的某届学生,一模二模校测都是前五十,然而考前松懈了,最后只600出头.\\
	\end{quote}

	\pause{}

	\begin{block}{The Markov's inequality \& Chebyshev's inequality}
		\[
			\Pr{\left(X\geqslant k\right)}\leqslant \frac{E(X)}{k}
			\qquad
			\Pr{\left(\left\vert X-\mu \right\vert \geqslant k\sigma\right)} \leqslant \frac{1}{k^2}
		\]
	\end{block}
	首先,大幅偏离均值本来就是极小概率事件, 今年的高考就算是没有一个黑马, 也是正常的.\\
	何况,偏离均值怎么就一定是RP爆发排名上升呢,还有与之对等的发挥爆炸直接垫底\ldots 当心黑马思维变成赌徒心态\\
\end{frame}


\begin{frame}
	\frametitle{平稳发挥\ Cont'd\qquad 行动节奏}
	怕考前状态调整不到最佳? 那就一直维持较好的状态,直到高考结束!
	\pause{}
	\begin{enumerate}
		\item 紧跟学校复习训练节奏. 请相信\, 四中的老师懂科学备考, 四中提供可靠的规划.
		\item 从现在开始,形成并维持自己的节奏: 在校的晚自习,居家的周末,考前的自由复习.
	\end{enumerate}
	\pause{}
  请利用好学习小组, 公开打卡, 互相激励. 跟不上时,主动向老师求助.
\end{frame}


\begin{frame}
	\frametitle{平稳发挥\ Cont'd\qquad 如何对待考试练习}
	\begin{block}{也许没参加过高考,但是肯定围观过运动会}
		如果每天训练只是跑八百, 运动会上一千米你绝对跑不完.
	\end{block}
	统练按照传统是没有任何监视的, 模拟考试也和高考差异不小. \\
	但请认真对待考试,当作是高考,甚至基于更高的重视.\\
	\vspace{0.3em} \pause{}

	检验\ 复习是否完备,心态是否稳定.\\
  关注\ 该拿到的,近期训练的,解答不够流畅的.\\
	\vspace{0.3em} \pause{}
	\begin{alertblock}{kind reminder}
		考试之后盯排名很没意思. 高考结束之前,比排名重要的事情有太多.
	\end{alertblock}
\end{frame}


\begin{frame}
	\frametitle{几个应该思考的问题}

\end{frame}
\begin{frame}
	\frametitle{关于认知和心态}
	\alert{please don't miss the highlight of the whole lecture}\\
	\vspace{2em}

	我是个接受过十二年基础教育,但假期作业没有做完过一次的人.\\
	2020届的高三又是在家里蹲中度过的.\\
	高考后,甚至留下不少打印的卷子可以留着解微分方程时当草稿纸.\\
	\vspace{0.3em}
	我应该如何让自己以积极的心态走进考场?我要如何做才能谈笑风生走出考场?\\
	\vspace{2em} \pause{}

	\begin{itemize}
		\item 尊重规律和极限
		\item 内心的平静
		\item 你仍然是完整的人
		\item 非竞争性的集体
	\end{itemize}
\end{frame}

\begin{frame}
	\frametitle{1\ 高考是宏观低速的,没有人可以突破经典力学}

	\begin{example}
		\sout{因为我不会PS所以是真的}小故事\\
		观摩北京大学科技创新与创业课答辩 \& 旁听China Theory Week前沿报告
	\end{example}

	\pause{}

	中国人常说``一分耕耘,一分收获'', 学过了物理学的能量守恒定律,化学的原子守恒.\\
	高考,大概也有某种守恒. 如果你没有做过相应的训练, 在考场上不可能凭空创造出满分答案.\\

	\pause{}
	\vspace{2em}

	\begin{itemize}
		\item 所以,请明确``你鸽掉的任务,都对高考产生影响'',不要欺骗自己.\\
		我的语文议论文,训练确实不足,也没有特殊技巧. 那么我不应该指望自己在那里拿到比他人更高的分数,\\
		所以我不必在考点疯狂记忆所谓抢分模板, 不必在答题时抓耳挠腮,不必交卷后辗转反侧.\\

		\item 但是,在最后一科交卷前都有进步可做,只是你需要压缩时间,有所取舍.\\
		我也许错过了老师讲解的某基因纯合致死+交叉互换的计算方式,\\
		但是我可以保留讲课时的题目,搜索相关的问题.尝试自己摸索规律,向同学询问推理过程,和老师答疑时进行订正.\\
	\end{itemize}
\end{frame}

\begin{frame}
	\frametitle{2\ ``保持良好(?)心态''}

	\begin{block}{减少干扰}
		``备考,六月,目不窥园.'' 没必要这样,但我还是要提醒诸位:\\
		学业已经够令人头大的了,不要再输入过多信息,别让你的大脑过载.\\
	\end{block}

	\pause{} \vspace{0.3em}

	排除外部干扰当然是重要的, 但是这还不够.\\
	首先你不是真的与世隔绝, 其次每天只看文化课的人也常有心态波动.\\
	\pause{}

	\begin{block}{说服自己}
		觉得自己继续不下去时,停下来,重新考量现状和理想之间的这条道路.\\
		用自己认可的理论,说服自己\ ``调整状态继续高效工作'',是可以做到的,是必须进行的.\\
	\end{block}
	\pause{}
	\begin{block}{\sout{遇到困难\ 睡大觉}}
		有些时候仅仅是你太累了,暂且咕掉今天的任务,休息一下吧.\\
		请务必防止任何器官,任何系统过载.
	\end{block}
	reminder:\quad 不必强迫自己一直维持某个状态走完高考,只要即使调整,保证状态适合复习,适合应试即可.\\
\end{frame}

\begin{frame}
	\frametitle{3\ 高考不是二向箔,你仍然是完整的人}
	高三的最后一个寒假,不到半年的复习时间,此诚危急存亡之秋也!\\

	\pause{}
	\begin{block}{\sout{我们是高考考生,不是高考人. 我们走上的是考场,不是刑场}}
		但是专注高考备考,不意味着高三只有高考备考.\\
		考生的生物属性,社会关系,人生价值,都还在.\\
		四中不会要求学生剥离其他生活要素只留下备考,你也没必要逼自己那么做.\\
	\end{block}

	和小组成员聊聊做题烦恼, 和舍友侃侃BTC投资, 和爸妈谈谈食堂吃了啥.
	这就是正常的生活,这就是高考考生应该有的生活.
	\vspace{0.3em}
	你是完整的人, 所以高考不会把你干掉.\\
	如果你做了对应目标分数的足够准备,就有能力且应当平稳地走完高考.\\
	\vspace{1em} \pause{}
\end{frame}

\begin{frame}
	\frametitle{4\ 你不是孤立的个体,融入这个群体}

	\begin{block}{换个想法}
		\[
			\begin{array}{ll}
			\text{我和小L都想去T大CS,我们要竞争}& \text{我们可以一起搬进紫荆公寓}\\
			\text{前两年没好好学语文,X老师不愿意帮我} & \text{常找老师帮忙,他不想均分难看}\\
			\text{我太菜了,问题没人帮忙} & \text{不少大佬喜欢费恩曼学习法}\\
			\text{学长又和我不熟,肯定不理我} &\text{学长会想帮大学拉人}
		\end{array}
		\]
	\end{block}
	\vspace{1em} \pause{}

	执行力不足,他们能够监督你. 行动力不足,他们能够鼓励你.\\
	题目不会做,讲解没看懂? 他们能够用你熟悉,你能接受的方式向你讲解.\\
	\vspace{1em}
	\sout{以及,小组不必讲什么贡献对等, 又不是AA制恰饭}
\end{frame}

\begin{frame}
	\frametitle{ending}

	\sout{我知道它不押韵因为我没上过语文课,而且这是高考前瞎写的}\\
	\sout{欢迎报考张江理工学院,我是hehelego.我们高考加油视频见}\\

	\begin{block}{做可能且可行的事情}
		尊重认知规律,\\
		尊重统计学规律.\\
		不要尝试突破生物学和物理学划定是上限.
	\end{block}

	\begin{block}{应试心态}
		放低期望, 拉高下限.\\
		稳住状态, 力争正常.
	\end{block}

	\begin{block}{关于未来}
		可观测宇宙足够广阔, 去发现深邃的未知,去追寻你认可的意义吧.
	\end{block}
\end{frame}

\end{document}
